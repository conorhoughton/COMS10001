%13_linked_lists.tex
%notes for the course PandA1 COMS10002 taught at the University of Bristol
%2017 Conor Houghton conor.houghton@bristol.ac.uk

%To the extent possible under law, the author has dedicated all copyright 
%and related and neighboring rights to these notes to the public domain 
%worldwide. These notes are distributed without any warranty. 

\documentclass[11pt,a4paper]{scrartcl}
\typearea{12}
\usepackage{graphicx}
\usepackage{pstricks}
\usepackage{listings}
\lstset{language=C}
\pagestyle{headings}
\markright{COMS10001 - PandA2 13\_linked\_lists - Conor}
\newcommand{\pter}{-$>$}

\begin{document}

\subsection*{13 - linked lists\footnote{\texttt{http://github.com/conorhoughton/COMS10001}}}

A data structure is a way of organizing and storing information on a
computer. Modern computer languages often have a selection of in-built
data structures, usually as a library. Here we will look at some of
the most common structures and implement them, by hand, in c. We will
see that different data structures are suited to different tasks. For
example, so far we have been using arrays in which the elements are
accessed by index, if you want the ith entry of a you write a[i] and
so access is $O(1)$. However, adding an element to the middle of an
array is hard since lots of elements have to be moved to make
room. This is illustrated in Table~\ref{table_reindexing}, adding an
entry is an $O(n)$ operation.

\begin{table}[b]
\begin{tabular}{cccc cccc}
0&1&2&3&4&5&6&7\\
a&b&c&e&f&g&h&\\
a&b&c&e&f&g&&h\\
a&b&c&e&f&&g&h\\
a&b&c&e&&f&g&h\\
a&b&c&&e&f&g&h\\
a&b&c&d&e&f&g&h
\end{tabular}
\caption{A schematic intended to illustrate the difficulty of adding the entry d to position 3 of an array, e-h first have to be moved to make room.\label{table_reindexing}}
\end{table}

Often, in fact, you don't need to access elements by index; you might
just need to go through them one-by-one in order but want to be able
to add extra elements in different locations. For this you can use a
linked list, which will be described below: accessing a specified
element is an $O(n)$ operation, but accessing each element one-by-one
in turn, is also $O(n)$ and adding an extra element is $O(1)$.

\subsubsection*{Linked list}

A linked list is a data structure in which each entry is stored in a
node which also stores the location of the next node. There is no
index, so it isn't possible to access the entry in the $i$th node
directly, in the way you can access the $i$th entry in an
array. However, since each node knows the location of the next one,
you can go through the nodes in turn.

For example, lets use an arrow to illustrate one node knowing the
location of the next, a linked list storing the entries $[3,1,9,10,4]$
would look like
\begin{equation}
3\rightarrow 1\rightarrow 9\rightarrow 10\rightarrow \times
\end{equation}
where the last node points to nothing, here illustrated with a
$\times$ and, in practice in c it is implemented as a NULL pointer. Now,
assuming the location of the first node, often called the \textsl{head},
is known, it can be accessed to find its entry, 3, and the location of
the next entry, indicated here by the arrow, this in turn can be
accessed to give the entry, 1, and the location of the next node. This
continues until the last node is identified by its NULL pointer. 

\begin{table}
\begin{lstlisting}[numbers=left]
struct node
{
  int entry;
  struct node *next;
};
\end{lstlisting}
\caption{A node. This is a node of a linked list for storing ints. It
  contains two variables, entry which stores the actual entry and
  next, which is a pointer to a node, this is the link. \label{c_node}}
\end{table}

\begin{table}
\begin{lstlisting}[numbers=left]
struct node * make_head(int new_entry)
{
  struct node * head = (struct node *)malloc(sizeof(struct node));
  head->entry=new_entry;
  head->next=NULL;
  return head;
}
\end{lstlisting}
\caption{A function for making the head. In line three a new node is
  created on the heap with a pointer to it called head. The entry for
  head is given the correct value, note that head is pointer, so its
  entry is head-$>$entry. \label{c_head}}
\end{table}


A struct for a node in a linked list is given in
Table~\ref{c_node}. When making a linked list you need first of all a
head, a node whose address you know. There is a function for making
the head in Table~\ref{c_head}. Now, after that the locations of the
subsequent nodes are the business of the linked list, the person using
the linked list only has to know the location of head, the location of
the other nodes is stored in the nodes themselves.


\begin{table}
\begin{lstlisting}[numbers=left]
void print_list(struct node * iterator)
{
  while(iterator->next!=NULL)
    {
      printf("%d\n",iterator->entry);
      iterator=iterator->next;
    }

  printf("%d\n",iterator->entry);
}
\end{lstlisting}
\caption{A function to print out the entries. Notice that when
  iterator-$>$next=NULL it doesn't print out the entry in iterator,
  that's why there are two printf statements.\label{c_print}}
\end{table}

\begin{table}
\begin{lstlisting}[numbers=left]
struct node * locate(struct node * iterator, int target)
{
  while(iterator->next!=NULL)
    {
      if(iterator->entry==target)
	return iterator;
      iterator=iterator->next;
    }
  if(iterator->entry==target)
    return iterator;

  return NULL;
}
\end{lstlisting}
\caption{A function to return the location of the node containing target. If target isn't found it returns NULL.\label{c_locate}}
\end{table}


Now, to iterate through the list is simply a matter of moving along
it. In Table~\ref{c_print} there is a function for printing out all
the elements. It starts at the head and then goes to head-$>$next and
keeps going until head-$>$next==NULL. Another iterator is given in
Table~\ref{c_locate}, this locates an entry and returns a pointer to
its node.


\begin{table}
\begin{lstlisting}[numbers=left]
void add_node(struct node * here, int new_entry)
{
  struct node * here_next=here->next;
  here->next = make_head(new_entry);
  here->next->next=here_next;
}
\end{lstlisting}
\caption{Add a node after
  \textbf{here}. \textbf{here\_next} stores the
  location of \textbf{here\pter next}, the new node is added at
  \textbf{here\pter next} and this new node's \textbf{next}, what is now
  \textbf{here\pter next\pter next}, is set to
  \textbf{here\_next}.\label{c_add}}
\end{table}

\begin{table}
\begin{lstlisting}[numbers=left]
void append_node(struct node * head, int new_entry)
{
  while(head->next!=NULL)
    head=head->next;
  head->next = add_head(new_entry);
}
\end{lstlisting}
\caption{Append a node. This goes to the end of the list and adds the
  new node there.\label{c_append}}
\end{table}

Adding an entry is mostly a matter of moving around the links. Say we
had the address of 9 in
\begin{equation}
3\rightarrow 1\rightarrow 9\rightarrow 10\rightarrow \times
\end{equation}
 and wanted to add a node containing 12 between 9 and 10. First we would store the address of 10 somewhere and add a new node at the end of the arrow from 9
\begin{equation}
3\rightarrow 1\rightarrow 9\rightarrow \_(\rightarrow)10\rightarrow \times
\end{equation}
where the $\rightarrow$ in brackets is to indicate that the location
of the node with 10 has been stored. Then, in the new node the entry
is set equal to 12 and the link is set pointing to the node with 10, giving
\begin{equation}
3\rightarrow 1\rightarrow 9\rightarrow 12\rightarrow 10\rightarrow \times
\end{equation}
Code to do this is in Table~\ref{c_add}, in Table~\ref{c_append} is
code for the related operation of adding a new node at the end. Deleting the node next to a node is done in a similar way, consider
\begin{equation}
3\rightarrow 1\rightarrow 9\rightarrow 10\rightarrow \times
\end{equation}
and say you wanted to delete the node after the one which stores 1. First you store the link to the node holding 10
\begin{equation}
3\rightarrow 1\rightarrow 9(\rightarrow) 10\rightarrow \times
\end{equation}
Then the node holding 9 is deleted
\begin{equation}
3\rightarrow 1\quad \not9(\rightarrow) 10\rightarrow \times
\end{equation}
and then you set the node with 1 to point to the one with 10
\begin{equation}
3\rightarrow 1\rightarrow 10\rightarrow \times
\end{equation}
\end{document}
