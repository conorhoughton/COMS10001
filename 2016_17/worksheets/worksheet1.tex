%worksheet1.tex
%problem set for the course PandA2 COMS10001 taught at the University of Bristol
%2017 Conor Houghton conor.houghton@bristol.ac.uk

%To the extent possible under law, the author has dedicated all copyright 
%and related and neighboring rights to these notes to the public domain 
%worldwide. These notes are distributed without any warranty. 



\documentclass[11pt,a4paper]{scrartcl}
\typearea{12}
\usepackage{graphicx}
\usepackage{pstricks}
\usepackage{listings}
\usepackage{color}

\newif\ifanswers
\answersfalse


\lstset{language=C}
\pagestyle{headings}
\markright{COMS10001 - PandA2 algorithms worksheet 1 - Conor}
\begin{document}

\subsection*{Algorithms Worksheet 1}

For each part of a question write the answer and include
workings. Each question is worth two marks, except Q3 which is worth
four, there is also two marks for attendance, but the number of marks
are capped at ten.

\begin{enumerate}

\item This question is about estimating the algorithmic complexity of
  evaluating a polynomial. Here, consider fixed
  sized variables, so multiplication and addition take roughly one
  step, irrespective of how many digits the number has. Once again,
  powers are calculated by multiplication.

\begin{enumerate}
\item What is the big-oh complexity of evaluating, that is finding the
  value of $p(x)$, of an order $n$ polynomial
$$p(x)=a_n x^n +a_{n-1}x^{n-1}+\ldots+a_1x+a_0$$
using straight-forward substitution?
\item Horner's method is a quicker method for evaluating a
  polynomial. If $x_o$ is the value that the polynomial needs to be
  evaluated on, let $b_n=a_n$ and then 
$$ b_{n-1}=a_{n-1}+x_o b_{n}$$
and
$$ b_{n-2}=a_{n-2}+x_0 b_{n-1}$$
right down to 
$$ b_0=a_0+x_0b_1$$ and $b_0=p(x_0)$ is the answer. What is the big-oh
complexity?
\end{enumerate}


\ifanswers 

\noindent Solution: So calculating $x^i$ is $i-1$ multiplications, there are
faster ways to work out powers, but you are told that they are
calculated by multiplication; multiplying by $a_i$ is one more, so
that is $i$ calculations, thus evaluating the polynomial is
$1+2+3+\ldots n$ multiplications, along with $n-1$ additions; thus
this is $\Theta(n^2)$. However, using Horner's method each $b_i$ is a few calculations and there are $n$ $b_i$s, so that means it is $\Theta(n)$.

\fi

\item This question is about the asymptotic behavior of
  different functions, in each case give big-Theta for $T(n)$; if
  $T(n)$ was the worst case run-time this would give big-Oh. There is
  no need to give any working for this problem. 

\begin{enumerate}

\item $T(n)=n^5+\frac{1}{n}+n(n-1)(n+2)^4$
\item $T(n)=n^2\log{n}+n^3$
\item $T(n)=2^n+n!$
\item $T(n)=\sum_{i=0}^ni$
\item $T(n)=\sqrt{n}n+n$
\item $T(n)=n^2/\log{n}+n$
\item $T(n)=(n^5+345n^4+36n)/(n^2+2n+1)$
\item $T(n)=1/(n^2+2n+1)$
\item $T(n)=[(n+1)(n+2)(n+3)]/[(n+4)(n+5)]$
\item $T(n)=n!/(n-1)!$
\end{enumerate}

\ifanswers 

\noindent Solution: So just take the leading term in $n$ each time

\begin{enumerate}

\item $\Theta(n^5)$
\item $\Theta(n^2\log{n})$
\item $\Theta(n!)$
\item $\Theta(n^2)$
\item $\Theta(\sqrt{n}n)$
\item $\Theta(n^2/\log{n})$
\item $\Theta(n^3)$
\item $\Theta(1)$
\item $\Theta(n)$
\item $\Theta(n)$
\end{enumerate}
\fi


\newpage

\item This question is about solving recursion relations using
  telescoping. In each case find the value of $T(n)$ by
  telescoping. Check your answer by substitution, it is permissible
  to combine these two steps by using telescoping to come up with an
  ansatz and then substituting it to fix values in the ansatz. Write down
  the big-Theta for the solution.

\begin{enumerate}
\item $T(n)=T(n-1)+3$ with $T(0)=1$
\item $T(n)=T(n-1)+3$ with $T(1)=1$
\item $T(n)=2T(n-1)+3$ with $T(0)=1$
\item $T(n)=3T(n-1)+2$ with $T(0)=1$
\end{enumerate}

\ifanswers

\noindent Solution: So 
$$T(n)=T(n-1)+3=T(n-2)+3+3=T(n-3)+3\cdot 3=\ldots$$ It would be easy
to solve this directly by telescoping, but lets use an ansatz, since there is clearly a $3$ for each iteration we'll try $T(n)=3n+A$, substituting in
$$3n+A=3n-3+A+3$$ so the equation holds for all $A$, alternatively substituting a more general
ansatz of the form $T(n)=Bn+A$ would give you
$$Bn+A=Bn-B+A+3$$ which holds for all $A$ and $B=3$. Either way
$T(n)=3n+A$, now the initial condition is $T(0)=1$ but setting $n=0$
gives $T(0)=A$ so $A=1$ and the solution is $T(n)=3n+1$. For the next
part the question is exactly the same except $T(1)=1$ so taking
$T(n)=3n+A$ again $T(1)=3+A=1$ and hence $A=-2$. For the next part
$$T(n)=2T(n-1)+3=4T(n-2)+2\cdot 3=2^3T(n-3)+(4+2+1)\cdots 3 =\cdots$$
Doing this directly by telescoping requires skill because you need to
know that $2^{n-1}+2^{n-2}+\ldots +1=2^n-1$, but you could guess
$$T(n)=2^nA+B$$ and hope for the best, substituting in gives
$$2^nA+B=2^nA+2B+3$$
so this is a solution when $B=-3$, hence
$$T(n)=A2^n-3$$
and the initial condition gives $T(0)=A-3=1$ so $A=4$ and
$$T(n)=2^{n+2}-3$$
Finally, for the last $T(n)=A3^n+B$ is obviously a good ansatz based on the previous example, now
$$A3^n+B=A3^n+3B+2$$
so $B=-1$ and the initial condition gives $A=2$, hence
$$T(n)=2\cdot 3^n-1$$
\fi

\item Use telescoping to guess an ansatz and then solve
  $T(n)=T(n-1)+3n$ with $T(1)=1$. What is the corresponding big-Theta?

\ifanswers

\noindent Solution: So telescoping gives a bit that looks like
$3n+3(n-1)+3(n-2)+\ldots$, hence a good ansatz is $T(n)=An^2+Bn+C$,
substituting that in gives
$$An^2+Bn+C=A(n-1)^2+B(n-1)+C+3n=An^2-2An+A+Bn-B+C+3n$$
or, after cancelling
$$-2An+A-B+3n=0$$
so $A=3/2$ and $B=A$. Thus
$$T(n)=\frac{3}{2}n^2-\frac{3}{2}n+C$$
and the initial condition means $C=1$.

\fi





\end{enumerate}

\end{document}
