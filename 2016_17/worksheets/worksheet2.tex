%worksheet2.tex
%problem set for the course PandA2 COMS10001 taught at the University of Bristol
%2017 Conor Houghton conor.houghton@bristol.ac.uk

%To the extent possible under law, the author has dedicated all copyright 
%and related and neighboring rights to these notes to the public domain 
%worldwide. These notes are distributed without any warranty. 



\documentclass[11pt,a4paper]{scrartcl}
\typearea{12}
\usepackage{graphicx}
\usepackage{pstricks}
\usepackage{listings}
\usepackage{color}

\newif\ifanswers
\answerstrue


\lstset{language=C}
\pagestyle{headings}
\markright{COMS10001 - PandA2 algorithms worksheet 2 - Conor}
\begin{document}

\subsection*{Algorithms Worksheet 2}

For each part of a question write the answer and include
workings. Each question is worth two marks, there are also two marks
for attendance.

\begin{enumerate}


\item Solve for $T(n)$ using the ansatz $T(n)=r^n$ for the following
  two step recursion relations. Solving for $r$ will give two values
  $r_1$ and $r_2$, this means that the general solution will be
  $T(n)=Ar_1^n+Br_2^n$. Use the two base values to find $A$ and $B$. 

\begin{enumerate}
\item $T(n)=2T(n-1)+3T(n-2)$ with $T(0)=0$ and $T(1)=4$.
\item $T(n)=T(n-2)$ with $T(0)=0$ and $T(1)=2$.
\end{enumerate}


\ifanswers 

\noindent Solution:

For (a) we have 
\begin{equation}
r^2=2r+3
\end{equation}
so $r^2-2r-3=0$ or $(r-3)(r+1)=0$ so
\begin{equation}
T(n)=3^nA+(-1)^nB
\end{equation}
and the initial conditions give $A+B=0$ and $3A-B=4$ so 
\begin{equation}
T(n)=3^n-(-1)^n
\end{equation}
For (b) we get $r^2=1$ so 
\begin{equation}
T(n)=A+(-1)^nB
\end{equation}
and the initial conditions give $A+B=0$ and $A-B=2$ so
\begin{equation}
T(n)=1-(-1)^n
\end{equation}
\fi

\item This question is about the master theorem. Use it to
  calculate big-Theta for $T(n)$ in each case. 

\begin{enumerate}
\item $T(n)= 25T(n/5)+4n^2$
\item $T(n)= 20T(n/5)+4n$
\item $T(n)= 16T(n/2)+2n^4$
\end{enumerate}

\ifanswers

\noindent Solutions: for the first one $\log_5{25}=2$ and $c=2$ so
this is the middle case and $T(n)\in \Theta(n^2\log n)$, for the
second $\log_5 20>1$ so it is the first case and $T(n)\in
\Theta(n^{\log_5{20}})$; the last one is in the middle case as well
since $\log_2{16}=4$ and $T(n)\in \Theta(n^4\log{n})$.

\fi

\item This question is about quicksort; use the quicksort algorithm to
  sort the set $\{4,7,8,10,1,2,5,9,3,6\}$ showing all your steps, use
  the median of triples on the first three entries to find the pivot;
  you don't need to go through the individual swaps involved in the
  in-place implementation, just divide the set around the pivot.

\ifanswers

\noindent Solution: for example

\begin{tabular}{cccccccccc}
4&7&8&10&1&2&5&9&3&6\\
4&1&2&5&3&6&\bf{7}&8&10&9\\
1&\bf{2}&4&5&3&6&\bf{7}&8&\bf{9}&10\\
1&\bf{2}&3&\bf{4}&5&6&\bf{7}&8&\bf{9}&10
\end{tabular}

\fi


\item This question is about quicksort in place; perform the first
  step of quicksort, dividing the set into two, on the set
  $\{4,7,8,10,1,2,5,9,3,6\}$ using the pivot $7$ and individual swaps.


\ifanswers

\noindent Solution: the bold numbers are being considered for swapping.

\begin{tabular}{cccccccccc}
4&7&8&10&1&2&5&9&3&6\\
\bf{4}&6&8&10&1&2&5&9&\bf{3}&7\\
4&\bf{6}&8&10&1&2&5&9&\bf{3}&7\\
4&6&\bf{8}&10&1&2&5&9&\bf{3}&7\\
4&6&\bf{3}&10&1&2&5&9&\bf{8}&7\\
4&6&3&\bf{10}&1&2&5&9&\bf{8}&7\\
4&6&3&\bf{10}&1&2&5&\bf{9}&8&7\\
4&6&3&\bf{10}&1&2&\bf{5}&9&8&7\\
4&6&3&\bf{5}&1&2&\bf{10}&9&8&7\\
4&6&3&\bf{5}&1&2&\bf{10}&9&8&7\\
4&6&3&5&\bf{1}&2&\bf{10}&9&8&7\\
4&6&3&5&1&\bf{2}&\bf{10}&9&8&7\\
4&6&3&5&1&2&7&9&8&10\\
\end{tabular}

\fi



\end{enumerate}

\end{document}
