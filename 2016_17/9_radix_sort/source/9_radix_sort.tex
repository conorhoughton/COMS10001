%9_radix_sort.tex
%notes for the course PandA2 COMS10001 taught at the University of Bristol
%2016-7 Conor Houghton conor.houghton@bristol.ac.uk

%To the extent possible under law, the author has dedicated all copyright 
%and related and neighboring rights to these notes to the public domain 
%worldwide. These notes are distributed without any warranty. 

\documentclass[11pt,a4paper]{scrartcl}
\typearea{12}
\usepackage{graphicx}
\usepackage{pstricks}
\usepackage{listings}
\lstset{language=C}
\pagestyle{headings}
\markright{COMS10001 - PandA2 9\_radix\_sort - Conor}
\begin{document}

\subsection*{9 - radix sort}

Radix sort is very old, older than electronic computing; it dates back
to sorting methods developed by Herman Hollerith in the 1880s to use
with punch cards and tabulating machines. It allowed the 1890 US
census to be counted in one year, the 1880 census had taken eight and
had been feared that the 1890 census wouldn't be counted in time for
the 1900 census. Hollerith went on to start a company, the Tabulating
Machine Company, which worked on censuses in lots of different
countries and was one of the four companies which merged to form what
became IBM. There are a number of varieties of radix sort, here we
will look at least significant digit sort.

Radix sort algorithms rely on the data being of the right form; for
our description we assume the data is made up of integer entries. The
algorithm is a bucketing algorithm, not a comparison one, as we have
had before, it involves making piles of entries according to some
property of the entry. These piles are called buckets, possibly
reflecting the way radix worked when it ran on punch cards and swept
them into buckets.

Least significant digit radix sort works by going through the array
one by one and putting the entries in ten buckets according to the
value of the last, or least significant, digit. The piles from each
bucket are then stacked back together from lowest to highest so the
entries are arranged in order of their last digit. The same process is
carried out on the next significant digit, the key thing is that
entries that go into the same bucket do so in the same order they were
in the array so the sorting of their least significant digit is
preserved. 

If the initial array was $\{27,17,23,14\}$ for example, then the first
run through will put them into three buckets, the {\bf 3} bucket will
hold 23, the {\bf 4} buckets 14 and the {\bf 7} bucket 27 and 17; when
the piles are put back in order the array is $\{23,14,27,17\}$. This
is then bucketed by the second digit, so the {\bf 1} bucket holds 14
and 17, in that order since that's the order they are in in the most
recent version of the array, the {\bf 2} bucket hold 23 and 27 so when
they are put together the array is sorted: $\{14,17,23,27\}$. This is
repeated until all the significant figures have been
sorted. Table~\ref{table_radix} gives an example.

While punch cards were literally put in buckets the usual way to radix
sort arrays is by counting. Basically, the algorithm runs through the
array twice. The first time through it counts how many entries there
are in each bucket, this allows the position each entry should have to
be worked out, on the second run the entries are put in the correct
place. An example of this is given in Table~\ref{table_buckets} and
code for radix sort on integers is given in Table~\ref{c_radix}.

As for complexity, for each significant figure there are $O(n)$ steps
as the algorithm runs through the list; in fact we know it runs though
the list a few times, once to calculate the bucket values, once to
place the entries in the new array and another time to copy the new
array back to the old one. It does this for each significant digit, if
there are $k$ significant digits this makes the algorithm $O(kn)$. We
don't treat the $k$ as a constant since it may depend on $k$ depending
on the nature of the array. For example, if the original array contain
values from one to $n$ that have been shuffled, the number of
significant digits in $n$ is $k=\log_{10}n$ so the algorithm is
$O(n\log{n})$. Conversely, maybe the array contains lots and lots of
repititions but only has entries up to some $n_0<<n$, then
$k=\log_{10}n_0$ is a constant and the run time is $O(n)$. Thus, radix
sort is extemely good in certain circumstances, but in general it is
outperformed by efficient comparison based algorithms.

\begin{table}
\begin{tabular}{c|cccccccccc}
&335&9383 &45 &9&886&2777&69&7793&383&386\\
0&9383&7793&383&335&45&886&386&2777&9&69\\
1&9&335&45&69&2777&9383&383&886&386&7793\\
2&9&45&69&335&9383&383&386&2777&7793&886\\
3&9&45&69&335&383&386&886&2777&7793&9383
\end{tabular}
\caption{Radix sort in action. In line 0 the units are sorted, in line
  1 the tens, in line 2 the hundred and in line 3 the thousands. The
  key point if that the order inside a bucket preserves the order that
  was already there, so when 45 and 69 both go in the {\bf 0} hundreds
  bucket, 45 is before 69 because they were already in that order in
  the previous line.\label{table_radix}}
\end{table}

\begin{table}
\begin{tabular}{c|cccccc|c|c}
 &  &  &  &  &  &  &{\bf 1}&{\bf 3}\\
 &  &  &  &  &  &  &&\\
a&21&53&63&41&61&23&3&3\\
\hline
b1&  &  &  &  &  & &3&6\\
b2&  &  &  &  &  &23&3&5\\
b3&  &  &61&  &  &23&2&5\\
b4&  &41&61&  &  &23&1&5\\
b5&  &41&61&  &63&23&1&4\\
b6&  &41&61&53&63&23&1&3\\
b7&21&41&61&53&63&23&0&3
\end{tabular}
\caption{The bucket in action. Here we have a list of numbers being
  sorted by the least significant digit, for illustrative purposes the
  entries all have least significant digit 1 or 3. The original array
  is marked \texttt{a}, the last two digits represent the {\bf 1} and
  {\bf 3} buckets, there are three entries in each bucket. The new
  array b is used to store the contents of the buckets, this is done
  in lines b1-b7. Since 1 comes before 3, the {\bf 1} bucket entries
  are assigned the first three slots in b and the {\bf 3} bucket the
  slots 4-6. Thus, the number for the {\bf 3} bucket is set to 6,
  which is the total number of entries in buckets for values less than
  or equal to 3. First 23 is considered in line b2, since its digit is
  3 it is in the {\bf 3} bucket. The number for the {\bf 3} bucket is
  6 so this entry is placed in position 6 and one is taken away from
  the number for the {\bf 3} bucket. In line b3 the 61 is considered,
  it is in the {\bf 1} bucket and the number for the {\bf 1} bucket is
  3, this is placed in slot 3 and the number for the {\bf 1} bucket is
  reduced to 2. This carries on until all the entries are placed in
  b. \label{table_buckets}}
\end{table}


\begin{table}
\begin{lstlisting}[numbers=left]
void radix_sort(int a[], int n, int a_bound)
{

   int exp_max=log10((double)a_bound)+1;
   int a_sort[n];
   int i,exp;

   for(exp=0;exp<exp_max;exp++){
      int buckets[10]={0};
      int digit=pow(10,exp);

      for(i=0;i<n;i++)
         buckets[a[i]/digit%10]++;

      for(i=1;i<10;i++)
         buckets[i]+=buckets[i-1];

      for(i=n-1;i>=0;i--){
	  buckets[a[i]/digit%10]--;
	  a_sort[buckets[a[i]/digit%10]]=a[i];
      }

      for(i=0;i<n;i++)
         a[i]=a_sort[i];

   }
}
\end{lstlisting}
\caption{Radix sort. Here a\_bound is the largest possible entry, this
  might be calculated by going through the entries and looking for the
  largest one, or it might known, this code is available as part of
  {\tt radix\_sort.c} and in this code the user specifies the bound on
  the random numbers, this bound is used to get a\_bound. A more
  sophisticated version of this program has a marker to prevent
  resorting of the elements that have fewer significant digits, this
  can be seen in {\tt radix\_sort\_better.c} in the github folder.\label{c_radix}}
\end{table}

\end{document}
