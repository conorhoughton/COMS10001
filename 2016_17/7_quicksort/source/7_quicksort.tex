%7_quicksort.tex
%notes for the course PandA1 COMS10001 taught at the University of Bristol
%2016 Conor Houghton conor.houghton@bristol.ac.uk

%To the extent possible under law, the author has dedicated all copyright 
%and related and neighboring rights to these notes to the public domain 
%worldwide. These notes are distributed without any warranty. 

\documentclass[11pt,a4paper]{scrartcl}
\typearea{12}
\usepackage{graphicx}
\usepackage{pstricks}
\usepackage{listings}
\lstset{language=C}
\pagestyle{headings}
\markright{COMS10002 - PandA1 7\_quicksort - Conor}
\begin{document}

\subsection*{7 Sorting - bubble sort and quicksort}

We have already looked at one sorting algorithm, insert sort and found
that it was $O(n^2)$. We also saw that it has only a $O(1)$ extra memory
requirement, that is, not including the memory to store the array
itself, it only required a fixed amount of memory to run. Here we are
going to look at other sorting algorithms, both because sorting
algorithms are important and useful and because they serve as a good
set of example algorithms for considering different aspects of
efficiency.

\subsubsection*{Bubble sort}

In bubble sort the list is sorted by going through it element by
element and swapping each element with the one next to it if they are
in the wrong order. This is repeated until the list is sorted. Thus,
the small numbers \lq{}rise to the top\rq{}, that is end up near the
start of the list, because they are swapped repeatedly with larger
numbers that start off above above them. This terminology can get
confusing since it can be hard to remember what you mean by
\lq{}top\rq{} and \lq{}bottom\rq{}, but the analogy is useful, the
elements move up or down the list with each successive run
through. Table \ref{table_bubble} gives an example of bubble sort in
action and code for bubble sort is given in Table~\ref{c_bubble}.

Bubble sort is $O(n^2)$, for the worst case where the list starts off
in reverse order it takes $n$ passes through the list to order it,
with each pass involving $n-1$ comparisons. This is because each pass
basically moves the largest unsorted element to its correct
position. Bubble sort is $O(1)$ in extra memory. However, although it
appears in these characteristics to match insert sort, it is regarded
as being inferior because, in practice, it is slower, particularly for
lists that are close to being sorted. There are improvements, like
cocktail sort, which alternates sorting big elements up and small ones
down, and comb sort which initially moves elements by more than one
step at a time. Really though, the appeal of bubble sort is its
simplicity and an intuitive appeal, since it is worse than insert
sort, in most cases, it isn't actually useful.

\begin{table}
\begin{tabular}{c|cccccc}
0&5&6&2&3&4&1\\
\hline
1&5&6&2&3&4&1\\
2&5&2&6&3&1&4\\
3&5&2&3&6&1&4\\
4&5&2&3&1&6&4\\
5&5&2&3&1&4&6\\
\hline
6&2&5&3&1&4&6\\
7&2&3&5&1&4&6\\
8&2&3&1&5&4&6\\
9&2&3&1&4&5&6\\
\hline
10&2&3&1&4&5&6\\
11&2&1&3&4&5&6\\
12&2&1&3&4&5&6\\
\hline
13&1&2&3&4&5&6\\
14&1&2&3&4&5&6\\
\hline
15&1&2&3&4&5&6\\
\end{tabular}
\caption{Bubble sort. Line zero is the original list and the
  horizontal lines seperate different runs. Each line represents a
  comparison and, when a swap is appropriate, a swap. It avoids
  comparisons at the end where the elements are already
  sorted.. \label{table_bubble}}
\end{table}

\begin{table}
\begin{lstlisting}[numbers=left]
void swap(int a[],int i, int j)
{
   int temp=a[i];
   a[i]=a[j];
   a[j]=temp;
}


void bubble(int a[], int n)
{
   int i,unfinished=1;
  
   while(unfinished){
      unfinished=0;
      for(i=0;i<n-1;i++)
         if(a[i]>a[i+1]){
	    unfinished=1;
	    swap(a,i,i+1);	    
	 }
   }
}
\end{lstlisting}
\caption{A bubble sort. The int unfinished is used to stop the while
  loop, at the start of each while loop it is set to zero, if any
  pairs need to be swapped it is changed to one; zero counts as false
  when evaluated as a boolean value, anything else casts to true. This
  pair of functions is part of {\tt bubble\_sort.c}. One easy way to
  improve this version is to avoid the unneeded comparisons with the
  elements that have already been sorted, this improved version is
  also $O(n^2)$ but is quicker, it can be seen at {\tt
    bubble\_sort\_better.c}. \label{c_bubble}}
\end{table}


\subsubsection*{Quicksort}

Quicksort, usually written like that, as one word, is probably the
most popular sorting algorithm. It was discovered in 1960 by Tony
Hoare. It works by choosing an element, possibly at random, called the
partition value or pivot, let's say it has value $p$. The list is then
split into two piles, elements less than $p$ and elements greater
than $p$. This is then repeated recursively on each of
the two piles with the recursion terminating if a pile has only one
element. Crucially, with a bit of care, the process of splitting the
list into two piles can be done in place, so no substantial extra
memory is needed. In reading the algorithm, most of the complexity
comes from the desire to sort in place, so keep in mind the central
idea, choose an element and make two piles, one for elements smaller
than the chosen value, one for elements larger, and then recurse onto
those two piles.

\begin{table}
\begin{tabular}{c|cccccc}
0&\underline{5}&6&2&3&4&\underline{1}\\
1&1&\underline{6}&2&3&\underline{4}&5\\
2&1&\underline{6}&2&\underline{3}&4&5\\
3&1&\underline{6}&\underline{2}&3&4&5\\
4&1&2&6&3&4&5\\
\end{tabular}
\caption{Dividing the piles in quicksort. Here the partition value is
  3 and the underlines mark the position of $i$ on the left and $j$ on
  the right. In line 0 the $i$ and $j$ are at the start and the end,
  since $5\ge 3$ and $1<3$ these values are swapped and $i$ and $j$
  moved. 6 needs to be moved, but $4\ge 3$ so it can't be, instead $j$
  is moved, next $3\ge 3$ so $j$ is decreased again, now, in line 3,
  $6\ge 3$ and $2<3$ so these entries are swapped. Increasing $i$ and
  decreasing $j$ would leave them in the wrong order, so the procedure
  is finished. The value of $i$ and $j$ in line 3 mark out the
  division between the low pile and the high
  pile.\label{table_quick_divide}}
\end{table}

So, let's examine quicksort in practise and ignore for now how the
partition value is chosen $p$. A marker $i$ is placed at the first
entry of the list, and another $j$ at the last entry. If $a[i]<p$ and
$a[j]\ge p$ then both $a[i]$ and $a[j]$ are in the correct pile and
nothing needs to be done. In this case $i$ is increased by one and $j$
is decreased. If $a[i]\ge p$ and $a[j]<p$ then both are in the wrong
pile and they are swapped, $i$ is then increased and $j$ decreased. If
$a[i]\ge p$ but $a[j]\ge p$ as well, then $a[i]$ needs to be swapped,
but can't be because $a[j]$ needs to stay put. In this case $j$ is
decreased by one; in the converse case where $a[i]$ and $a[j]$ are
both less than $p$, $i$ is increased by one. This process of swapping
finishes when $i\ge j$. In practise that means $i$ is increased until
$a[i]\ge p$ and $j$ is decreased until $a[j]<p$, then, if $i<j$ the
two entries are swapped. This part of quicksort is illustrated in
Table~\ref{table_quick_divide}.

The algorithm is then applied recursively to each of the piles. If a
pile has only one entry it is already sorted, if it has two it can be
sorted with a single comparison and, if needed, a single swap. We still
haven't dealt with the choice of the partition point. Obviously the
best thing is to choose a value that splits the pile more-or-less in
half each time; if you are very unlucky and choose the lowest value
then the lower pile would be empty. The normal approach is called
median of threes, three entries are chosen and the middle one is used
as the partition point. This means that the lower pile will have at
least one element so the algorithm is guaranteed to converge and, by
looking at three it makes it likely that it will give a reasonably
even split.

Before finally looking at the code for quicksort, we need to think a
little bit about how to be careful about keeping track of where the split
is between the two piles. Furthermore, it is worthwhile not wasting what
we know about the partition value: its value lies between the two
piles. Thus, at the end we can swap it to the point where the two
piles join and then not include it in either since it is already in
the right place. A collection of functions for performing quicksort
are given in Table~\ref{c_quick_extras} and
Table~\ref{c_quick}. Table~\ref{table_quick_careful} illustrates
the part of the precise algorithm that divides the data into two piles
that is used in the code.

The run time for this algorithm needs to be worked out recursively. If
there are $n$ entries $T(n)$ requires the approximately $n$
comparisons and swaps needed to partition the algorithm. It also calls
quicksort on the two piles afterwards. If these piles are roughly
equal then
\begin{equation}
T(n)\approx cn+2T(n/2)
\end{equation}
where a more exact version might look something like $T(n)=c_1+c_2n +
T(n_1)+T(n_2)$ where $n_1+n_2=n$, but the approximate version is good
enough to find the algorithmic complexity provided we assume
$n_1\approx n_2\approx n/2$. 

We will have a look at how we might solve this recursion relation, but
we are in the happy position of not needing to do this, we can just
apply the master theorem. The master theorem is used to calculate the algorithmic behavior of solutions to the recursion
\begin{equation}
T(n)\approx f(n)+aT(n/b)
\end{equation}
which is our situation with $a=b=2$ and $f(n)=cn$. This means $f(n)\in
\Theta(n)$ which is of the form $\Theta(n^c)$ with $c=1$, now we also
have $\log_ba=\log_22=1$ so this is the marginal case where $T(n)\in
\Theta(n^c\log n)$, with $c=1$ this means $T(n)\in \Theta(n\log n)$,
with the warning that this applies to the $T(n)$ that satisfies the
recursion relation, which isn't always all the run times, in
particular, here, the run time only satisfies the recursion if the
pivot divides the piles in half.

Since we were able to use the master theorem there is really no need
to think further about how to solve the recursion relation. We will do
so anyway for nosing around reasons. Now to solve the recursion
relation consider
\begin{equation}
T(n)=An\log{n}
\end{equation}
so
\begin{equation}
2T(n/2)=2A\frac{n}{2}\log{\frac{n}{2}}=An\log{n}-An
\end{equation}
and so this solves the approximate equation when $A=c$. In fact by substitution you can see that
\begin{equation}
T(n)=An\log{n}+Cn
\end{equation}
solves the equation for any value of $C$, if we were solving this
equation exactly then the initial value would be used to fix $C$; here
though we are only interested in the large $n$ behavior, so this isn't
important.

Of course, the problem here is that we solved the equation by roughly
knowing the answer and checking our guess was correct. You can see
that you might arrive at this guess by staring at the equation and
trying lots of things. Another, more straight forward approach, would
be to do a few steps of the iteration
\begin{eqnarray}
T(n)&=&cn+2T(n/2)\\
    &=&cn+2c(n/2)+4T(n/4)\\
    &=&2cn+4c(n/4)+8T(n/8)\\
    &=&3cn+8c(n/8)+16T(n/16)\\
    &=&\ldots\\
    &=&rcn+2^rT(n/2^r)
\end{eqnarray}
which carries on until $n/2^r=1$, or $r=\log{n}$. Substituting that into the equation we get 
\begin{equation}
T(n)=cn\log{n}+2^{\log{n}}T(1)=cn\log{n}+nT(1)
\end{equation}
or $T(n)\in O(n\log{n})$ as before. 

\begin{table}
\begin{lstlisting}[numbers=left]
int median(int a[],int i, int j, int k)
{

  if(a[i]>a[j]&&a[i]>a[k])
    return (a[j]>a[k]) ? j : k;
  if(a[i]<a[j]&&a[i]<a[k])
    return (a[j]>a[k]) ? k : j;

  return i;
}

void swap(int a[],int i, int j)
{
  int temp=a[i];
  a[i]=a[j];
  a[j]=temp;
}
\end{lstlisting}
\caption{Some functions for quicksort. These are two functions needed
  for quicksort, it has been split into two parts to help it fit
  nicer, the other part contains the actual algorithm, this part
  contains two of the functions it needs, the swap basically swaps the
  values at a[i] and a[j] and, if you unpack all the ternary
  operators, median returns i, j or k depending on which of a[i], a[j]
  and a[k] has the value in the middle when they are put in
  order.\label{c_quick_extras}}
\end{table}

\begin{table}
\begin{lstlisting}[numbers=left]
void quick_r(int a[],int first, int last)
{
   if(first>=last)
      return;
   if(last==first+1)
      if(a[first]<a[last])
         return;
      else{
	swap(a,first,last);
	return;
      }

   int i=first,j=last-1;

   swap(a,median(a,first,first+1,last),last);
  
   while(i<j){
      while(a[j]>=a[last]&&j>first)
         j--;
      while(a[i]<a[last])
         i++;
      if(i<j)
	 swap(a,i,j);
   }
   swap(a,last,i);

   quick_r(a,first,i-1);
   quick_r(a,i+1,last);
}

void quick(int a[], int n)
{
   quick_r(a,0,n-1);
}
\end{lstlisting}
\caption{Quicksort. This is the business part of the quicksort
  algorithm, see Table~\ref{c_quick_extras} for some of the functions
  used. Notice how $j$ is decreased first and is made to stop if it
  reaches first, i is then increased and stops if a[i]$<$a[last], this
  means that, at the end, when i$\ge$j, i gives the first entry of the
  upper pile. Since the partition value has been placed for safe
  keeping at the end of the upper pile, it can be swapped for this
  value. This is illustrated in
  Table~\ref{table_quick_careful}.\label{c_quick}}
\end{table}



\begin{table}
\begin{tabular}{c|cccccc}
0&6&3&2&5&4&1\\
1&6&1&2&5&4&3\\
2&$6_i$&1&2&5&$4_j$&3\\
3&$6_i$&1&2&$5_j$&4&3\\
4&$6_i$&1&$2_j$&5&4&3\\
5&$2_i$&1&$6_j$&5&4&3\\
6&$2_i$&$1_j$&6&5&4&3\\
7&2&$1_j$&$6_i$&5&4&3\\
8&2&1&3&5&4&6
\end{tabular}
\caption{The division method for our implementation of quicksort. This
  shows the location of \texttt{i} and \texttt{j} as quicksort is running with a partition value of 3. In line 7 \texttt{i}$>$\texttt{j}
  so the algorithm stops and in line 8 the partition value is swapped
  back to the place marked by \texttt{i}.\label{table_quick_careful}}
\end{table}

We have been assuming that the group is divided roughly in two at each
step. In fact, this depends on a good choice of the partition
value. If the partition value is chosen to be the lowest value in the
list, the pile of entries with value lower than the partition value
will be empty. If we have adapted the strategy of excluding the
partition value from both piles, then this run through will have
reduced the number of elements by one. If this were to happen every
time the run time for quicksort would be $O(n^2)$, as bad as insertion
sort since each run through involves $n$ comparisons and reduces the
number of elements by only one. In other words, the worst case run
time for quicksort is $O(n^2)$. If a median of threes approach is used
the partition value can't be the lowest value, but it could be the
second lowest and if the second lowest value is chosen each time, the
algorithm is still $O(n^2)$. However, this is hugely unlikely, for
most initial data the chance of getting $O(n^2)$ behavior is extremely
small. An exception might be, for example, a set where there are lots
of equal small entries with a few large entries mixed in.

Quicksort is probably the most used sorting algorithm. Although its
worst case behavior is $O(n^2)$ it usually runs at $O(n\log{n})$ and,
in most applications, it usually runs faster than other $O(n\log{n})$
algorithms. Furthermore, it is done in place, it doesn't require
substantial amounts of extra memory.

\end{document}
