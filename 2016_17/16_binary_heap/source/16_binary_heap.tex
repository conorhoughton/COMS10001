%16_binary_heap.tex
%notes for the course PandA1 COMS10002 taught at the University of Bristol
%2017 Conor Houghton conor.houghton@bristol.ac.uk

%To the extent possible under law, the author has dedicated all copyright 
%and related and neighboring rights to these notes to the public domain 
%worldwide. These notes are distributed without any warranty. 

\documentclass[11pt,a4paper]{scrartcl}
\typearea{12}
\usepackage{graphicx}
\usepackage{pstricks}
\usepackage{listings}
\usepackage{tikz-qtree}
\lstset{language=C}
\pagestyle{headings}
\markright{COMS10001 - PandA2 16\_binary\_heap - Conor}

\begin{document}
\tikzset{every tree node/.style={minimum width=2em,draw,circle},
         blank/.style={draw=none},
         edge from parent/.style=
         {draw,->, edge from parent path={(\tikzparentnode) -- (\tikzchildnode)}},
         level distance=1.5cm}

\subsection*{16 - binary heap DRAFT}

A binary heap is a data structure which allows elements to be easily
added and removed while keeping track of the largest element. They are
important because they are used to implement priority queues in
scheduling, for example, in a router. We have also seen another
potential application, in Dijskra's algorithm the lowest distance node
is always evaluated next and a binary heap could be used to track
which node is the lowest. Here, for definiteness, we will consider a
tree designed to keep track of the highest node, rather than the
lowest as in the the Dijstra example, but, of course, this isn't a
significant difference.

A binary heap is a complete binary tree; since it is a binary tree all
the nodes have up to two child nodes, a complete tree is one where all
the layers except the lowest layer are full. A binary heap also the
heap property which states that every node is smaller than, or equal
to, its parent.















\begin{figure}
   \begin{center}
\begin{tikzpicture}
\Tree [.50 [.20 \edge[]; {10} \edge[]; {25} ] [.70 \edge[blank];
    \node[blank]{}; \edge[]; [.80 \edge[]; {75} \edge[]; {95} ] ] ]
\end{tikzpicture}
\end{center}
\caption{An example binary tree for looking at heights. The node 50
  has height four, the distance down to 75 and 95. The node 20 has
  height two and the node 70 has height
  three.\label{fig_example_tree}}
\end{figure}


\end{document}
