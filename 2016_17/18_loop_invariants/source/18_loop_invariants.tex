%18_loop_invariants.tex
%notes for the course PandA2 COMS10001 taught at the University of Bristol
%2016 Conor Houghton conor.houghton@bristol.ac.uk

%To the extent possible under law, the author has dedicated all copyright 
%and related and neighboring rights to these notes to the public domain 
%worldwide. These notes are distributed without any warranty. 

\documentclass[11pt,a4paper]{scrartcl}
\typearea{12}
\usepackage{amsmath}
\usepackage{graphicx}
\usepackage{pstricks}
\usepackage{listings}
\usepackage{tikz}
\usetikzlibrary{calc,positioning}
\usetikzlibrary{shapes,arrows}
\lstset{language=C}
\pagestyle{headings}
\markright{COMS10001 - PandA2 13\_loop\_invariants - Conor}

\begin{document}

\subsection*{18 - loop invariants}

Loop invariants are a tool for verify algorithms, in a sense they
present an algorithm as a recursion in the mathematical sense and make
it ameanable to proof. It is easiest to approach the loop invariant
through an example, the function in Table~\ref{c_insertsort} does
insert sort. Now consider line 14; at this point the subarray of
\texttt{a} running from 0 to \texttt{i}, which we will call
\texttt{a[0:i]}, contains the 0 to \texttt{i} elements of the original
list, but sorted. This is the \textsl{loop invariant}, it is a
statement that is true before the first loop, at the end of each
iteration and the fact it is true at the end means the algorithm has
achieved its goal. By checking loops for invariants we can be
confident the loop will achieve its goal.

\begin{table}
\begin{lstlisting}[numbers=left]
void sort(int a[],int n)
{
   int i,j,this_a;

   for(i=1;i<n;i++){
      this_a=a[i];
      j=i-1;

      while(j>=0&&this_a<a[j]){
         a[j+1]=a[j];
	 j=j-1;
      }
      a[j+1]=this_a;
      //a[0 . . . i] contains the first i+1 elements, sorted
   }
}
\end{lstlisting}
\caption{Insert sort, as seen in 1\_introduction.\label{c_insertsort}}
\end{table}

Thus a loop invariant is a statement that has three properties:
initialization, it is true at the start; maintenance, if it is true at
the start of an iteration, it is true at the end and utility, when the
loop terminates the loop invariant shows that the algorithm has served
its purpose. For insertion sort the sorted list is \texttt{a[0]} at
the start, it is trivially sorted and contains \texttt{i=0} element of
list. If \texttt{a[0:i-1]} is sorted and contains the first $i$
elements, then the \texttt{i} iteration adds the $i+1$ element and
moves it down the list so that it is bigger than the element to its
left and larger than the one to its right. This means that the new,
larger, list is sorted, maintaining the invariant. Finally, when the
loop terminates $i=n-1$ so the invariant tells us $a[0:n-1]$ contains
all the elements in the origin list, sorted. Hence, this is a list
invariant.

As another example, consider binary search; A code listing is given in
Table~\ref{c_binary_search}. In it a sorted array \texttt{a} is
searched for a key \texttt{val}, there are two markers \texttt{low}
and \texttt{high} and at each iteration the maker \texttt{mid} is set
to the midpoint of \texttt{low} and \texttt{high}, if
\texttt{a[mid]==val} it terminates, if not it exploits the fact the
list is sorted and sets \texttt{low=mid+1} if \texttt{a[mid]} is less
than \texttt{val} or sets \texttt{high=mid-1} otherwise. The key point
is that it changes the markers so that is \texttt{val} is in the
original list it is in \texttt{a[low:high]}. This gives us the loop
invariant which is that if \texttt{val} is in the list it is in
\texttt{a[low:high]}. This is true at the start when
\texttt{a[low:high]} is the whole list, if it is true at the start of
each iteration it is true at the end because the list is sorted and,
finally, the loop terminates when it contains only one element, since, either,
this is \texttt{val} or \texttt{val} wasn't in the list, then
the algorithm has served its purpose.

\begin{table}
\begin{lstlisting}[numbers=left] 
int search(int a[],int n, int val)
{
  int mid, low=0, high=n-1;

  while(low<high){
      mid=(low+high)/2;
      if(a[mid]==val){
        low=mid;
        high=mid;
      }
      else if(val>a[mid])
	low=mid+1;
      else
	high=mid -1;
    }
  if(a[high]==val)
    return high;
  return -1;
}
\end{lstlisting}
\caption{Binary search, this is modified from 3\_search.\label{c_binary_search}}
\end{table}



\end{document}
